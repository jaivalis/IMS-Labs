\documentclass[11pt]{article}
\usepackage{array}
\usepackage{tabularx}
\usepackage{graphicx}
\usepackage{algorithm}
\usepackage{algorithmic}
\usepackage{pgfplotstable}
\usepackage{pgfplots}
\usepackage{filecontents}
\usepackage{amsmath}



\title{
	\textbf{IMS Assignment 4}
}

\author{Tobias Stahl \\ 10528199 \and Ioannis Giounous Aivalis \\ 10524851 }

\begin{document}

\maketitle

\section{Introduction}
This report is about Assignment 4 of the UvA course Intelligent Multimedia Systems. The goal of the first part of Assignment 4 is to implement functions to align images using the SIFT \footnote{http://
www.vlfeat.org/} implementation and implement a RANSAC algorithm.\\
The second part is about stitching two images together.

\section{Exercise 1}
In this part of the assignment six images of a boat from different angles are used to affine them to each other. Using the SIFT implementation the matching regions between two images are found.\\
Applying the RANSAC algorithm, the best transformation parameters are found. The process to do that is as follows:

\begin{itemize}
	\item 	Repeat N times
	\item	Randomly select three matches of all matches
	\item	Compute transformation matrix
	\item	Transform all regions in matches
	\item	Count inliers
	\item	If there are more inliers with this setting than with the previous best setting, keep the 			matrix
	\item	After all iterations are done, transform the image with the best transformation matrix
\end{itemize}

The subset of three regions out of the set of matches is chosen, since this is the minimal number of regions needed to 


\section{Conclusion}
In this assignment the functions to create a Gaussian filter and its derivatives, and to convolve those with any image were implemented and tested with different exercises. 




\end{document}